\section{Limitations}
\label{sec:limitations}
\subsection{Limitations from conservative datasets.} We only considered a small subset of code snippets in Java available on Stackoverflow. 
As a result the 8 insecure patterns we have considered are tailored to this small subset of code snippets. 
A more rich dataset would have allowed to capture more naunce insecure patterns.Moreover, we only analzed code snippets with Java/Andrioid tags.
If we have analyzed code snippets from other popular languages~\cite{stackoverflow-survey}, we would have been able to
discover more insecure patterns.   


\subsection{Limitations from program repair techniques.} The program repair techniques in section~\ref{subsec:code-repair}, we have applied on erronous code snippets, 
before converting them to Jimple IR, are simple semi-automated simple parsing  repairs. It would have been better to see if state-of-the-art  
the program repair techniques can be adjusted to repair them. Especially automated program repair tools which can apply quick single edit fixes (e.g, Google's OSS-Fuzz and Microsoft's Springfield project) are quite enticing to apply for fixing erronous code snippets as discussed in this great survey paper~\cite{automated-program-repair}.



\noindent
\label{limitatons:jimple} 
\subsection{Limitations due to PPA/Jimple grammer.} The partial program analysis tool (PPA)~\cite{dagenais2008enabling} we have used theto run 
analysis depends on Jimple.
Jimple, a 3-address IR, was designed to handle, and simplify several
difficulties while performing optimizations on the stack-based Java bytecode directly~\cite{vallee1998jimple}. However, 
the simple grammer of Jimple on which we are running backword flow analysis can not handle annonymous class. 
Annonymous classes are the most common way developers declare the X509TrustManager interface (insecure pattern \# 3). Because of
Jimple inability to express annonymous classes, we can not detect the insecure pattern \#3.

%The 

\noindent
\subsection{Limitations from disregarding comments.} Insecurity in production level code can also come from 
insecure advices given in text forms on Stackoverflow. 
As we are only considering code snippets we are missing out on them. 
Ideally we would like to use NLP techniques to detect the insecure advices. Also some code snippets do contains insecure patterns 
but the developers has warned against blindly copy pasting this insecure code snippets as comments in the code snippets. As we are disregarding
comments, it is quite debateable that considering the code snippets would make sense at any one before copy pasting this
code snippet would read the comment and already know abut the insecurity of the code snippet.     