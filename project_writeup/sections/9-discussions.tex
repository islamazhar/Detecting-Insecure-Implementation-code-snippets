
\section{Discussions on ML based detection techniques}
\label{sec:discussion}
%\subsection{ML based detection techniques.} 
Identification of insecure patterns, vulnerabilities, and code-smells using ML based 
techniques have been gaining tractions recently. This is mainly for two reasons 
i) leveraging the wealth of open source code available ii) adaptive neural network models 
which can capture/represent the complex patterns of source code. While givng a overview of ongoing  research in this area is outside
the scope of the paper, we can mention some work exciting research work closely related to ours that uses ML techniques. 
For example, Zhou et al. proposed~\cite{devign_neurips19} vulnerability detection model ``Devign'' for code snippets in C language. 
Their key idea to detect vulnerability is by capturing the abstract syntax tree structure of the source code via Gated-Graph Neural Network (GGNN).
The study in~\cite{Automated-Vulnerability-Detection-in-Source-Code-Using-Deep-Representation-Learning} proposed automated vulnerability detection 
tools using deep feature representation learning. 
%Interesting the study in~\cite{} hinted at detecting the issue of subjectivity the machine learning models suffers for detectin code smells by replicating the work in~\cite{}.

One problem associated with using machine learning models is that either they are function level detection 
model~\cite{Automated-Vulnerability-Detection-in-Source-Code-Using-Deep-Representation-Learning}, 
do have limited ability 
to reason why source code is vulnerable~\cite{devign_neurips19}, or suffers from subjectivity~\cite{are-we-there-yet}.


Static analysis based approaches such as ours, can overcome for two reasons i) it can reason about the source code, and ii) as insecure patterns are repetitive. As a result we can build a static analysis tool by observing 
the common insecure trends to reason about the  source. 
We agree this can be an overstretched claim, and leave it as a future work this paper.  
%\paragraph{Sythesis}


\iffalse
\subsection{ML based detection techniques}
\subsection{Code Repair}
\subsection{Limitations for failing to construct a CFG:} program repair are basically simple edits
PPA is very old

The tension between soundness and completeness: 
Our tool is the analysis part but not complete. but keyword based analysis also should introduce no FP. 
Synthesis
Other Sources of vulnerability: 
Lack of training /Knowledge: Vulnerable Online Turorials 
Lack of screening tools:
Being short and focusing on a working solution while writing code snippets.
\fi