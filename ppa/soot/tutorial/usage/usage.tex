
\documentclass{article}
\usepackage{fullpage}
\usepackage{html}
\usepackage{longtable}

\title{Soot command-line options}
\author{Patrick Lam (\htmladdnormallink{plam@sable.mcgill.ca}{mailto:plam@sable.mcgill.ca})\\
Feng Qian (\htmladdnormallink{fqian@sable.mcgill.ca}{mailto:fqian@sable.mcgill.ca})\\
Ond\v{r}ej Lhot\'ak (\htmladdnormallink{olhotak@sable.mcgill.ca}{mailto:olhotak@sable.mcgill.ca})\\
John Jorgensen\\
}

\renewcommand{\arraystretch}{1.3}

\begin{document}

\maketitle

\tableofcontents

\section{SYNOPSIS}

Soot is invoked as follows:
\begin{quote}
{\tt java} {\it javaOptions} {\tt soot.Main} [ {\it sootOption}* ] {\it classname}*
\end{quote}

\section{DESCRIPTION} 
This manual documents the command line options of the Soot
bytecode compiler/optimizer tool. In essence, it tells you what you can
use to replace the {\it sootOption} placeholder which appears in the {\sc
SYNOPSIS}.



\par

The descriptions of Soot options talk about three categories of
classes: argument classes, application classes, and library classes.

\par

{\it Argument classes} are those you specify explicitly to
Soot. When you use Soot's command line interface, argument
classes are those classes which are either listed explicitly on
the command line or found in a directory specified with the
{\tt -process-dir} option. When you use the Soot's Eclipse
plug-in, argument classes are those which you selected before
starting Soot from the Navigator popup menu, or all classes in
the current project if you started Soot from the Project
menu.

\par

{\it Application classes} are classes that Soot analyzes,
transforms, and turns into output files.

\par

{\it Library classes} are classes which are referred to,
directly or indirectly, by the application classes, but which are
not themselves application classes. Soot resolves these classes
and reads {\tt .class} or {\tt .jimple} source files for
them, but it does not perform transformations on library classes
or write output files for them.

\par

All argument classes are necessarily application classes.  When
Soot is not in ``application mode'', argument
classes are the only application classes; other classes
referenced from the argument classes become library classes.

\par

When Soot is in application mode, every class
referenced from the argument classes, directly or indirectly, is
also an application class, unless its package name indicates that
it is part of the standard Java runtime system.

\par

Users may fine-tune the designation of application and library
classes using the Application Mode Options.


\par

Here is a simple example to clarify things. Suppose your program
consists of three class files generated from the following
source:

\begin{quote}\begin{verbatim}

// UI.java
interface UI {
  public void display(String msg);
}

// HelloWorld.java
class HelloWorld {
  public static void main(String[] arg) {
    UI ui = new TextUI();
    ui.display("Hello World");
  }
}

// TextUI.java
import java.io.*;
class TextUI implements UI {
  public void display(String msg) {
      System.out.println(msg);
  }
}

\end{verbatim}\end{quote}


\par

If you run 

\begin{quote}\begin{verbatim}

java soot.Main HelloWorld

\end{verbatim}\end{quote}

{\tt HelloWorld} is the only argument class and the only
application class.  {\tt UI} and {\tt TextUI} are library
classes, along with {\tt java.lang.System},
{\tt java.lang.String}, {\tt java.io.PrintStream}, and a
host of other classes from the Java runtime system that get
dragged in indirectly by the references to {\tt String} and
{\tt System.out}.

\par

If you run 

\begin{quote}\begin{verbatim}

java soot.Main --app HelloWorld

\end{verbatim}\end{quote}

{\tt HelloWorld} remains the
only argument class, but the application classes include {\tt UI}
and {\tt TextUI} as well as {\tt HelloWorld}.
{\tt java.lang.System} et. al. remain library classes.


\par

If you run 

\begin{quote}\begin{verbatim}

java soot.Main -i java. --app HelloWorld

\end{verbatim}\end{quote}

{\tt HelloWorld} is still the only argument class, but the set
of application classes includes the referenced Java runtime
classes in packages whose names start with {\tt java.} as well
as {\tt HelloWorld}, {\tt UI}, and {\tt textUI}. The set
of library classes includes the referenced classes from other
packages in the Java runtime.


\section{OPTIONS}


\subsection{General Options}


\begin{description}

  \item[
  {\tt -h}, 
  {\tt -help}]

Display the textual help message and exit immediately without
further processing.



  \item[
  {\tt -pl}, 
  {\tt -phase-list}]

Print a list of the available phases and sub-phases, then exit.



  \item[
  {\tt -ph}{ \it phase}, 
  {\tt -phase-help}{ \it phase}]

Print a help message about the phase or sub-phase named
{\it phase}, then exit.  To see the help message of
more than one phase, specify multiple phase-help options.



  \item[
  {\tt -version}]

Display information about the version of Soot being run, then
exit without further processing.



  \item[
  {\tt -v}, 
  {\tt -verbose}]

Provide detailed information about what Soot is doing as it runs.



  \item[
  {\tt -interactive-mode}]

Runs interactively, with Soot providing detailed information as it iterates through intra-procedural analyses.



  \item[
  {\tt -app}]


\par

Run in application mode, processing all classes referenced by
argument classes.



  \item[
  {\tt -w}, 
  {\tt -whole-program}]


\par

Run in whole program mode, taking into consideration the whole
program when performing analyses and transformations. Soot
uses the Call Graph Constructor to build a call graph for the
program, then applies enabled transformations in the Whole-Jimple
Transformation, Whole-Jimple Optimization, and Whole-Jimple
Annotation packs before applying enabled intraprocedural
transformations.

\par

Note that the Whole-Jimple Optimization pack is normally disabled
(and thus not applied by whole program mode), unless you also
specify the Whole Program Optimize option.



  \item[
  {\tt -ws}, 
  {\tt -whole-shimple}]


\par

Run in whole shimple mode, taking into consideration the whole program
when performing Shimple analyses and transformations. Soot uses the
Call Graph Constructor to build a call graph for the program, then
applies enabled transformations in the Whole-Shimple Transformation
and Whole-Shimple Optimization before applying enabled intraprocedural
transformations.

\par

Note that the Whole-Shimple Optimization pack is normally disabled
(and thus not applied by whole shimple mode), unless you also
specify the Whole Program Optimize option.
      


  \item[
  {\tt -validate}]

Causes internal checks to be done on bodies in the various Soot IRs,
to make sure the transformations have not done something strange.
This option may degrade Soot's performance.



  \item[
  {\tt -debug}]

Print various debugging information as Soot runs, particularly
from the Baf Body Phase and the Jimple Annotation Pack Phase.



  \item[
  {\tt -debug-resolver}]

Print debugging information about class resolving.



\end{description}


\subsection{Input Options}


\begin{description}

  \item[
  {\tt -cp}{ \it path}, 
  {\tt -soot-class-path}{ \it path}, 
  {\tt -soot-classpath}{ \it path}]


\par

Use {\it path} as the list of directories in which Soot
should search for classes. {\it path} should be a series of
directories, separated by the path separator character for your
system.

\par

If no classpath is set on the command line, but the system
property {\tt soot.class.path} has been set, Soot uses its
value as the classpath.

\par

If neither the command line nor the system properties specify a
Soot classpath, Soot falls back on a default classpath consisting
of the value of the system property {\tt java.class.path}
followed {\it java.home}{\tt /lib/rt.jar}, where
{\it java.home} stands for the contents of the system property
{\tt java.home} and {\tt /} stands for the system file
separator.



  \item[
  {\tt -process-dir}{ \it dir}]


\par

Add all classes found in {\it dir} to the set of argument classes
which is analyzed and transformed by Soot.  You can specify the
option more than once, to add argument classes from multiple directories.


\par

If subdirectories of {\it dir} contain {\tt .class} or
{\tt .jimple} files, Soot assumes that the subdirectory names
correspond to components of the classes' package names. If
{\it dir} contains {\tt subA/subB/MyClass.class}, for
instance, then Soot assumes {\tt MyClass} is in package
{\tt subA.subB}.



  \item[
  {\tt -ast-metrics}]

			If this flag is set and soot converts java to jimple then AST metrics will be computed.
	


  \item[
  {\tt -src-prec}{ \it format}]

(default value: {\tt c})

Sets {\it format} as Soot's preference for the type of source files to read when
it looks for a class.




Possible values:\\
\begin{longtable}{p{1in}p{4in}}
{\tt c}, {\tt class} 
&

Try to resolve classes first from {\tt .class} files found in
the Soot classpath.  Fall back to {\tt .jimple} files
only when unable to find a {\tt .class} file.
\\
{\tt only-class} 
&

Try to resolve classes first from {\tt .class} files found in
the Soot classpath.  Do not try any other types of files even when
unable to find a {\tt .class} file.
\\
{\tt J}, {\tt jimple} 
&

Try to resolve classes first from {\tt .jimple} files found in
the Soot classpath.  Fall back to {\tt .class} files only when
unable to find a {\tt .jimple} file.
\\
{\tt java} 
&

Try to resolve classes first from {\tt .java} files found in
the Soot classpath.  Fall back to {\tt .class} files only when
unable to find a {\tt .java} file.
\\

\end{longtable}


  \item[
  {\tt -full-resolver}]

Normally, Soot resolves only that application classes and any classes that they
refer to, along with any classes it needs for the Jimple typing, but it does not
transitively resolve references in these additional classes that were resolved
only because they were referenced. This switch forces full transitive resolution
of all references found in all classes that are resolved, regardless of why they
were resolved.

In whole-program mode, class resolution is always fully transitive. Therefore,
in whole-program mode, this switch has no effect, and class resolution is
always performed as if it were turned on.



  \item[
  {\tt -allow-phantom-refs}]

Allow Soot to process a class even if it cannot find all classes
referenced by that class. This may cause Soot to produce
incorrect results.



  \item[
  {\tt -use-old-type-assigner}]

(default value: {\tt false})

Disable the reimplemented type assigner. This uses improved data 
structures to provide a significant performance benefit, but has
not yet been thoroughly tested.



  \item[
  {\tt -main-class}{ \it class}]


\par

By default, the first class mentioned on the command-line is treated
as the main class (entry point) in whole-program analysis. This option
overrides this default.




\end{description}


\subsection{Output Options}


\begin{description}

  \item[
  {\tt -d}{ \it dir}, 
  {\tt -output-dir}{ \it dir}]

(default value: {\tt ./sootOutput})

Store output files in {\it dir}. {\it dir} may be
relative to the working directory.



  \item[
  {\tt -f}{ \it format}, 
  {\tt -output-format}{ \it format}]

(default value: {\tt c})


\par

Specify the format of output files Soot should produce, if
any.

\par

Note that while the abbreviated formats ({\tt jimp},
{\tt shimp}, {\tt b}, and {\tt grimp}) are easier to
read than their unabbreviated counterparts ({\tt jimple},
{\tt shimple}, {\tt baf}, and {\tt grimple}), they may
contain ambiguities. Method signatures in the abbreviated
formats, for instance, are not uniquely determined.




Possible values:\\
\begin{longtable}{p{1in}p{4in}}
{\tt J}, {\tt jimple} 
&

Produce {\tt .jimple} files, which contain a textual
form of Soot's Jimple internal representation.
\\
{\tt j}, {\tt jimp} 
&

Produce {\tt .jimp} files, which contain an abbreviated form
of Jimple.
\\
{\tt S}, {\tt shimple} 
&

Produce {\tt .shimple files}, containing a textual form of
Soot's SSA Shimple internal representation.  Shimple adds
Phi nodes to Jimple.
\\
{\tt s}, {\tt shimp} 
&

Produce .shimp files, which contain an abbreviated form of
Shimple.
\\
{\tt B}, {\tt baf} 
&

Produce {\tt .baf} files, which contain a textual form of
Soot's Baf internal representation.
\\
{\tt b} 
&

Produce {\tt .b} files, which contain an abbreviated form of Baf. 
\\
{\tt G}, {\tt grimple} 
&

Produce {\tt .grimple} files, which contain a textual
form of Soot's Grimp internal representation.
\\
{\tt g}, {\tt grimp} 
&

Produce {\tt .grimp} files, which contain an abbreviated form
of Grimp.
\\
{\tt X}, {\tt xml} 
&

Produce {\tt .xml} files containing an annotated
version of the Soot's Jimple internal representation.
\\
{\tt n}, {\tt none} 
&

Produce no output files.
\\
{\tt jasmin} 
&

Produce {\tt .jasmin} files, suitable as input to the jasmin
bytecode assembler.
\\
{\tt c}, {\tt class} 
&

Produce Java {\tt .class} files, executable by any Java
Virtual Machine.
\\
{\tt d}, {\tt dava} 
&

Produce {\tt .java} files generated by the Dava decompiler.
\\

\end{longtable}


  \item[
  {\tt -outjar}, 
  {\tt -output-jar}]

Saves output files into a Jar file instead of a directory. The output
Jar file name should be specified using the Output Directory
({\tt output-dir}) option. Note that if the output Jar file exists
before Soot runs, any files inside it will first be removed.



  \item[
  {\tt -xml-attributes}]

Save in XML format a variety of tags which Soot has attached to
its internal representations of the application classes. The XML
file can then be read by the Soot plug-in for the Eclipse IDE,
which can display the annotations together with the program
source, to aid program understanding.



  \item[
  {\tt -print-tags}, 
  {\tt -print-tags-in-output}]

Print in output files (either in Jimple or Dave) a variety of tags which
Soot has attached to
its internal representations of the application classes. The tags will
be printed on the line succeeding the stmt that they are attached to.



  \item[
  {\tt -no-output-source-file-attribute}]

Don't output Source File Attribute when producing class files.



  \item[
  {\tt -no-output-inner-classes-attribute}]

            Don't output inner classes attribute in class files.



  \item[
  {\tt -dump-body}{ \it phaseName}]


\par

Specify that {\it phaseName} is one of the phases to be dumped.
For example -dump-body jb -dump-body jb.a would dump each
method before and after the jb and jb.a
phases.  The pseudo phase name ``ALL''
causes all phases to be dumped.

\par

Output files appear in subdirectories under the
soot output directory, with names like
{\it className}/{\it methodSignature}/{\it phasename}-{\it graphType}-{\it number}.in
and
{\it className}/{\it methodSignature}/{\it phasename}-{\it graphType}-{\it number}.out.
The ``in'' and
``out'' suffixes distinguish the internal
representations of the method before and after the phase
executed.



  \item[
  {\tt -dump-cfg}{ \it phaseName}]


\par

Specify that any control flow graphs constructed during the
{\it phaseName} phases should be dumped.
For example -dump-cfg jb -dump-cfg bb.lso would dump
all
CFGs constructed during the jb and bb.lso
phases.  The pseudo phase name ``ALL''
causes CFGs constructed in all phases to be dumped.

\par
The control flow graphs are dumped in the form
of a file containing input to dot graph visualization
tool.  Output dot files are stored beneath the soot
output directory, in files with names like:
{\it className}/{\it methodSignature}/{\it phasename}-{\it graphType}-{\it number}.dot,
where {\it number} serves to distinguish graphs in phases
that produce more than one (for example, the Aggregator may
produce multiple ExceptionalUnitGraphs).



  \item[
  {\tt -show-exception-dests}]

(default value: {\tt true})

Indicate whether to show exception destination edges as
well as control flow edges in
dumps of exceptional control flow graphs.



  \item[
  {\tt -gzip}]

(default value: {\tt false})

This option causes Soot to compress output files of intermediate representations
with GZip. It does not apply to class files output by Soot.



\end{description}


\subsection{Processing Options}


\begin{description}

  \item[
  {\tt -p}{ \it phase opt:val}, 
  {\tt -phase-option}{ \it phase opt:val}]


\par

Set {\it phase}'s run-time option named {\it opt} to
{\it value}.

\par

This is a mechanism for specifying phase-specific options to
different parts of Soot. See {\it Soot phase options} for
details about the available phases and options.



  \item[
  {\tt -O}, 
  {\tt -optimize}]

Perform intraprocedural optimizations on the application classes.



  \item[
  {\tt -W}, 
  {\tt -whole-optimize}]

Perform whole program optimizations on the application
classes. This enables the Whole-Jimple Optimization pack as well
as whole program mode and intraprocedural optimizations.



  \item[
  {\tt -via-grimp}]

Convert Jimple to bytecode via the Grimp intermediate
representation instead of via the Baf intermediate
representation.



  \item[
  {\tt -via-shimple}]

Enable Shimple, Soot's SSA representation. This generates Shimple
bodies for the application classes, optionally transforms them
with analyses that run on SSA form, then turns them back into
Jimple for processing by the rest of Soot. For more information,
see the documentation for the {\tt shimp}, {\tt stp}, and
{\tt sop} phases.



  \item[
  {\tt -throw-analysis}{ \it arg}]

(default value: {\tt pedantic})

This option specifies how to estimate the exceptions which each statement
may throw when constructing exceptional CFGs.
        



Possible values:\\
\begin{longtable}{p{1in}p{4in}}
{\tt pedantic} 
&

Says that any instruction may throw any
Throwable whatsoever. Strictly speaking this is
correct, since the Java libraries include the
Thread.stop(Throwable)
method, which allows other threads to cause arbitrary exceptions
to occur at arbitrary points in the execution of a victim thread.
\\
{\tt unit} 
&

Says that each statement in the intermediate representation
may throw those exception types associated with the corresponding
Java bytecode instructions in the JVM Specification.  The
analysis deals with each statement in isolation, without regard
to the surrounding program.
\\

\end{longtable}


  \item[
  {\tt -omit-excepting-unit-edges}]


\par

When constructing an ExceptionalUnitGraph or
ExceptionalBlockGraph, include edges to an exception
handler only from the predecessors of an instruction which may
throw an exception to the handler, and not from the excepting
instruction itself, unless the excepting instruction has
potential side effects.



\par

Omitting edges from excepting units allows more accurate flow
analyses (since if an instruction without side effects throws an
exception, it has not changed the state of the computation). This
accuracy, though, could lead optimizations to generate
unverifiable code, since the dataflow analyses performed by
bytecode verifiers might include paths to exception handlers from
all protected instructions, regardless of whether the
instructions have side effects.  (In practice, the pedantic throw
analysis suffices to pass verification in all VMs tested with
Soot to date, but the JVM specification does allow for less
discriminating verifiers which would reject some code that might
be generated using the pedantic throw analysis without also
adding edges from all excepting units.)



  \item[
  {\tt -trim-cfgs}]


\par

When constructing CFGs which include exceptional edges, minimize
the number of edges leading to exception handlers by analyzing
which instructions might actually be executed before an exception
is thrown, instead of assuming that every instruction protected
by a handler has the potential to throw an exception the handler
catches.


\par

-trim-cfgs is shorthand for -throw-analysis
unit -omit-excepting-unit-edges -p jb.tt enabled:true.



\end{description}


\subsection{Application Mode Options}


\begin{description}

  \item[
  {\tt -i}{ \it pkg}, 
  {\tt -include}{ \it pkg}]


\par

Designate classes in packages whose names begin with
{\it pkg} (e.g. {\tt java.util.}) as application
classes which should be analyzed and output. This option allows
you to selectively analyze classes in some packages that Soot
normally treats as library classes.

\par

You can use the include option multiple times, to designate
the classes of multiple packages as application classes.

\par

If you specify both include and exclude options, first the
classes from all excluded packages are marked as library classes,
then the classes from all included packages are marked as
application classes.



  \item[
  {\tt -x}{ \it pkg}, 
  {\tt -exclude}{ \it pkg}]


\par

Excludes any classes in packages whose names begin with
{\it pkg} from the set of application classes which are
analyzed and output, treating them as library classes instead.
This option allows you to selectively exclude classes
which would normally be treated as application classes


\par

You can use the exclude option multiple times, to designate
the classes of multiple packages as library classes.

\par

If you specify both include and exclude options, first the
classes from all excluded packages are marked as library classes,
then the classes from all included packages are marked as
application classes.



  \item[
  {\tt -include-all}]

Soot uses a default list of packages (such as java.) which are deemed
to contain library classes. This switch removes the default packages from the
list of packages containing library classes.
Individual packages can then be added using the exclude option.



  \item[
  {\tt -dynamic-class}{ \it class}]


\par

Mark {\it class} as a class which the application may load
dynamically. Soot will read it as a library class even if it is
not referenced from the argument classes.  This permits whole
program optimizations on programs which load classes dynamically
if the set of classes that can be loaded is known at compile
time.

\par

You can use the dynamic class option multiple
times to specify more than one dynamic class.



  \item[
  {\tt -dynamic-dir}{ \it dir}]


\par
Mark all class files in {\it dir} as classes that may
be loaded dynamically. Soot will read them as library classes
even if they are not referenced from the argument classes. 

\par

You can specify more than one directory of potentially dynamic
classes by specifying multiple dynamic directory options.



  \item[
  {\tt -dynamic-package}{ \it pkg}]


\par

Marks all class files belonging to the package {\it pkg}
or any of its subpackages as classes which the application may
load dynamically. Soot will read all classes in {\it pkg} as
library classes, even if they are not referenced by any of the
argument classes. 

\par
To specify more than one dynamic package, use the dynamic
package option multiple times.



\end{description}


\subsection{Input Attribute Options}


\begin{description}

  \item[
  {\tt -keep-line-number}]

Preserve line number tables for class files throughout the
transformations.



  \item[
  {\tt -keep-bytecode-offset}, 
  {\tt -keep-offset}]
Maintain bytecode offset tables for class files throughout the transformations.


\end{description}


\subsection{Annotation Options}


\begin{description}

  \item[
  {\tt -annot-purity}]

Purity anaysis implemented by Antoine Mine and based on the paper
A Combined Pointer and Purity Analysis Java Programs
by Alexandru Salcianu and Martin Rinard.



  \item[
  {\tt -annot-nullpointer}]

Perform a static analysis of which dereferenced pointers may have
null values, and annotate class files with attributes
encoding the results of the analysis.  For details, see the 
documentation for Null Pointer Annotation and for the 
Array Bounds and Null Pointer Check Tag Aggregator.



  \item[
  {\tt -annot-arraybounds}]

Perform a static analysis of which array bounds checks may safely
be eliminated and annotate output class files with attributes
encoding the results of the analysis. For details, see the
documentation for Array Bounds Annotation and for the Array
Bounds and Null Pointer Check Tag Aggregator.



  \item[
  {\tt -annot-side-effect}]

Enable the generation of side-effect attributes.



  \item[
  {\tt -annot-fieldrw}]
Enable the generation of field read/write attributes.


\end{description}


\subsection{Miscellaneous Options}


\begin{description}

  \item[
  {\tt -time}]

Report the time required to perform some of Soot's transformations.



  \item[
  {\tt -subtract-gc}]

Attempt to subtract time spent in garbage collection from
the reports of times required for transformations.



\end{description}



\end{document}

